\chapter{Summary}
In conclusion, all main goals of this thesis were accomplished successful. It gave us a detailed know-how about the construction of a simple virtual machine and its working flow. It is also mentionable the way how simple source codes or generally assembler instructions are being translated into machine readable languages. Besides, it was also explained how to use formal languages and grammars in syntax analyses.

In this thesis a graphical representation of a virtual machine was developed to simplify its overall handling and to give users a comprehensible overview of program work flows. Also a structured visualisation of data memory is given where stack frames and expression stacks can be inspected.

A main emphasis was laid to enable students debugging NoBeard programs. One of the most interesting topic was to investigate how real debugger work and reproducing it by using classical observer pattern.

Hopefully this thesis will support the courses of {\em Theoretical Informatics} held for future students of HTBLA Leonding to get an easier viewpoint in the field of compiler construction and system programming.